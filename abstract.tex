%!TEX root = main.tex
%%%%%%%%%%%%%%%%%%%%%%%%%%%%%%%%%%%%%%%%%%%%%%%%%%%%%%%%%%%%%%%%%%%%%%%%%%%%%%%%%%%%%%%%%%%%%%%%%%%%%%
%
%   Filename    : abstract.tex 
%
%   Description : This file will contain your abstract.
%                 
%%%%%%%%%%%%%%%%%%%%%%%%%%%%%%%%%%%%%%%%%%%%%%%%%%%%%%%%%%%%%%%%%%%%%%%%%%%%%%%%%%%%%%%%%%%%%%%%%%%%%%

\begin{abstract}

\begin{comment}
From 150 to 200 words of short, direct and complete sentences, the abstract 
should be informative enough to serve as a substitute for reading the thesis document 
itself.  It states the rationale and the objectives of the research.  

In the final thesis document (i.e., the document you'll submit for your final thesis defense), the 
abstract should also contain a description of your research results, findings, 
and contribution(s).

%
%  Do not put citations or quotes in the abract.
%

Keywords can be found at \url{http://www.acm.org/about/class/class/2012?pageIndex=0}.  Click the 
link ``HTML'' in the paragraph that starts with ''The \textbf{full CCS classification tree}...''.
\end{comment}

This study presents how an interaction was designed that led towards introducing balance in the work of musicians across all stages in music composition. Observation and user research led to having a deeper understanding of the various needs, gains and pain points musicians encounter when composing. An iterative process of design and development was continuously employed which led to improving the interaction design within the prototype. The processes described in this paper show how insights were uncovered from a comprehensive set of usability tests and inspections done. These insights led to the development of a more usable and acceptable musical composition tool as seen from the results in the final tests.

%Musical composition is a delicate and disciplined art form that is tedious and repetitive. It involves three main activities: ideation, sketching, and revision. Given that musical composition is a creative process, composers sometimes receive ideas while outside and would need an effective tool to write them down. This study explores the design of an interaction that aims to balance the work of composers with the help of a mobile application. To create an application that follows the creative process of composers, an iterative software engineering was employed. Several usability tests using different tools and setups were done to gain insight on their musical composition process and improve the prototype. Results were positive with regards to the interaction design of the resulting application. Users generally liked its simplicity and straightforwardness, especially the selection interaction. However, the application still lacks some musical notation modifiers and hence needs more improvement and redesign of the interface.


%Certain compositional tasks such as figuring out succeeding notes often requires trial-and-error. Existing technology has employed musical metacreation to assist in this process. This endows machines with the artificial creative capacity to perform musical tasks. In review, the existing technology has not been generally used in all stages of the musical composition process. By combining several interaction technologies, composers can benefit by being able to do their tasks with significantly less cognitive load and time.

\begin{flushleft}
\begin{tabular}{lp{4.25in}}
\hspace{-0.5em}\textbf{Keywords:}\hspace{0.25em} & Human Centric Computing, Human Computer Interaction, Interaction Design, User Interface Design, Interaction Techniques, Gestural Input, Sound and music computing, Computational creativity, Usability testing \\
\end{tabular}
\end{flushleft}
\end{abstract}
