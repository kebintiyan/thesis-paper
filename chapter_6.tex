%!TEX root = main.tex

\chapter{Conclusion, Recommendations, and Future Work}

	This chapter will discuss the conclusion of the study, recommendations for improvement, and possible future work.

	\section{Conclusion}
		\subsection{Interaction Design}

			This study provides a framework for designing mobile musical composition applications. This was done through user research which included performing interviews with composers and observing their creative processes. The results of the user research led to the design and development of a usable mobile musical composition tool. This was improved and redesigned repetitively over the course of several iterations. Each iteration involved testing with composers and gathering quantitative and qualitative feedback, analyzing the results, and improving the application for the next iteration. 

			The user tests led to a deeper understanding of the composers' musical composition process and their way of thinking when using musical notation applications. Initial assumptions were also proven wrong like in the case of Komp's handwriting input which, although took inspiration from the traditional method of writing on music sheets, did not work well for the composers. A lot of changes were also made in the application to follow the assumptions and expectations of composers. 

			One important change came from the results of iteration 1: the highlight interaction. Since people would normally scroll using a one-finger drag/swipe on mobile, the highlight interaction was delegated to a two-finger drag/swipe. It was surprising to find out that the composers could not figure out the highlight interaction during the iteration 1 testing. When the scroll and highlight interaction were switched, the composers became less confused and were able to use the application more naturally. 

			Another useful finding was on how composers expected the accidentals menu to work. At first it was set up to follow music theory, but it was found that composers expected it to work like bold or underline in word processors. This caused them to think that the buttons did not work or was unresponsive. Other important changes that were implemented were the separate transpose menu, the placing of the menus and buttons, and the music playback visualization. 

			The resulting musical composition application provided composers with the necessary space and tools to perform basic musical composition tasks. Adding, editing, and deleting were the core features of the application and so were assigned to simple gestures. Adding was done through the combination of the notation menu, the arrow keys, and the cursor. Editing and deleting needed revisions in the interaction to make it more natural for the composers. More advanced tasks were also made possible through the use of gestures. Users could highlight to easily perform actions like cut, copy, or paste as well as transposition, and retrograde-inversion.  

		\subsection{User Testing and Evaluation}

			The results of the user tests provided quantitative and qualitative data on the user experience of the application. In iteration 1, only the core features were implemented and tested (see Figure \ref{tab:results-features-it1}). Most of the problems in that iteration occurred due to the highlight interaction. This was fixed for iteration 2 by switching the gesture for highlight with the scroll gesture. 

			For iteration 2, the transpose interaction was hard to figure out because it also used the same arrow keys used to move the cursor. In iteration 3 the transpose interaction was given its own arrow keys in a separate menu (see Figure \ref{fig:transpose}) that only appears when there are highlighted notes. Similarly, the time and key signature menu needed improvements. The main issue with the menu was that it used sliders which made selection imprecise. An improved menu with buttons for common time signatures and a radial menu for the key signatures was designed as shown in Figure \ref{fig:time-key-signature}.

			Changes from the user testing results from iteration 3 mostly involved the placement of items in the editor as well as changes to the playback. The top menu was moved to the bottom so that users would only have to focus on one area for menu items (see Figure \ref{fig:before-after-menu}). The piano button was also moved beside the notation menu since some composers mentioned that they could not find it at the bottom menu (see Figure \ref{fig:before-after-menu}). The playback was also improved to allow starting from the measure where the cursor is currently pointing at so that it would be easier for composers to quickly listen to a specific measure they are working on. The current note/rest that was being played was shown to guide composers (see Figure \ref{fig:before-after-playback}).

			The main feedback from the composers in iteration 4 was that Flow lacked in terms of features and modifiers. Other than that, comments were surprisingly positive. A lot liked the cursor and how it made adding and selecting immediately obvious. Others also stated that the menus were well organized and the other features easy to find. 

			There was a common group of testers that were part of the testing from iteration 1 - 3. This was done to observe if there was any improvement in the application for them. Going back to Table \ref{tab:summarized-common-samples} and Figure \ref{fig:common-samples-line}, it can be seen that there was a continuous improvement in the overall interaction and user experience for the composers in the common samples group. The greatest growth came from iteration 1 to iteration 2, due to the change in the highlight interaction. With regard to the features, the feature that raised in score the highest was the delete feature. 

			In the two-way t-test performed with the common samples against the isolates from iteration 3, there is a significant difference in both cases. This suggests that the tests were relevant. When the common samples are compared with the control group from iteration 4, the result is that they do not have any significant difference. However, when only the iteration 3 scores from the common samples are compared with the control group, the two-way t-test results show significant difference. 


	\section{Recommendations and Future Work}

		As with any application, improvements can always be made. As mentioned by most of the composers, the application would be better with more features and musical notation modifiers. Future work could incorporate these features and modifiers in the application. Tests would still be needed in this case to ensure that the add features do not make the experience harder for composers. Further testing with more composers is also needed to gather more feedback and continue improving the application. 

		The results of the initial tests also led to the design and development of a usable mobile musical composition tool that aided composers through musical metacreation. Given that only an initial prototype was used during the testing, the musical metacreation feature was not implemented in the prototype. However, the interviews and tests done during the early prototyping stage suggest that musical metacreation, or being given suggestions on possible notes to write next, would be a valuable tool for composers during their \textit{ideation} activities. Future work could include the integration of a machine learning inference engine that might possibly help composers in the events of a ``creative block''.




