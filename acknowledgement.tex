%!TEX root = main.tex
%%%%%%%%%%%%%%%%%%%%%%%%%%%%%%%%%%%%%%%%%%%%%%%%%%%%%%%%%%%%%%%%%%%%%%%%%%%%%%%%%%%%%%%%%%%%%%%%%%%%%%
%
%   Filename    : abstract.tex 
%
%   Description : This file will contain your abstract.
%                 
%%%%%%%%%%%%%%%%%%%%%%%%%%%%%%%%%%%%%%%%%%%%%%%%%%%%%%%%%%%%%%%%%%%%%%%%%%%%%%%%%%%%%%%%%%%%%%%%%%%%%%

\topskip0pt
\vspace*{\fill}
\begin{center}
	\textbf{Acknowledgements}

	The authors would like to thank Dr. Arnulfo Azcarraga, Mr. Ryan Dimaunahan, and Mr. Juan Carlo Magsalin for sharing their time and expertise. 
\end{center}
\vspace*{\fill}

%Musical composition is a delicate and disciplined art form that is tedious and repetitive. It involves three main activities: ideation, sketching, and revision. Given that musical composition is a creative process, composers sometimes receive ideas while outside and would need an effective tool to write them down. This study explores the design of an interaction that aims to balance the work of composers with the help of a mobile application. To create an application that follows the creative process of composers, an iterative software engineering was employed. Several usability tests using different tools and setups were done to gain insight on their musical composition process and improve the prototype. Results were positive with regards to the interaction design of the resulting application. Users generally liked its simplicity and straightforwardness, especially the selection interaction. However, the application still lacks some musical notation modifiers and hence needs more improvement and redesign of the interface.


%Certain compositional tasks such as figuring out succeeding notes often requires trial-and-error. Existing technology has employed musical metacreation to assist in this process. This endows machines with the artificial creative capacity to perform musical tasks. In review, the existing technology has not been generally used in all stages of the musical composition process. By combining several interaction technologies, composers can benefit by being able to do their tasks with significantly less cognitive load and time.